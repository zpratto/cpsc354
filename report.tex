\documentclass{article}

\usepackage{tikz} 
\usetikzlibrary{automata, positioning, arrows} 

\usepackage{amsthm}
\usepackage{amsfonts}
\usepackage{amsmath}
\usepackage{amssymb}
\usepackage{fullpage}
\usepackage{color}
\usepackage{parskip}
\usepackage{hyperref}
  \hypersetup{
    colorlinks = true,
    urlcolor = blue,       % color of external links using \href
    linkcolor= blue,       % color of internal links 
    citecolor= blue,       % color of links to bibliography
    filecolor= blue,        % color of file links
    }
    
\usepackage{listings}
\usepackage[utf8]{inputenc}                                                    
\usepackage[T1]{fontenc}                                                       

\definecolor{dkgreen}{rgb}{0,0.6,0}
\definecolor{gray}{rgb}{0.5,0.5,0.5}
\definecolor{mauve}{rgb}{0.58,0,0.82}

\lstset{frame=tb,
  language=haskell,
  aboveskip=3mm,
  belowskip=3mm,
  showstringspaces=false,
  columns=flexible,
  basicstyle={\small\ttfamily},
  numbers=none,
  numberstyle=\tiny\color{gray},
  keywordstyle=\color{blue},
  commentstyle=\color{dkgreen},
  stringstyle=\color{mauve},
  breaklines=true,
  breakatwhitespace=true,
  tabsize=3
}

\theoremstyle{plain} 
   \newtheorem{theorem}{Theorem}[section]
   \newtheorem{corollary}[theorem]{Corollary}
   \newtheorem{lemma}[theorem]{Lemma}
   \newtheorem{proposition}[theorem]{Proposition}
\theoremstyle{definition}
   \newtheorem{definition}[theorem]{Definition}
   \newtheorem{example}[theorem]{Example}
\theoremstyle{remark}    
  \newtheorem{remark}[theorem]{Remark}

\title{CPSC-354 Report}
\author{Zach Pratto  \\ Chapman University}

\date{\today} 

\begin{document}

\maketitle

\begin{abstract}
This is the place to write an abstract. Not much has been abstracted yet.
\end{abstract}

\setcounter{tocdepth}{3}
\tableofcontents

\section{Introduction}\label{intro}

\section{Week by Week}\label{homework}

\subsection{Week 1}

Week 1 aligns with the first week of the semester. 

\subsubsection{Notes and Exploration}

This section is optional. 

\subsubsection{Homework}

The MU puzzle from the book asks us to transform MI into MU or $MI \;\Rightarrow\; MU$ using
only the following rules:

\begin{quote}
\textbf{Rule 1}:
If you possess a string whose last letter is I, you can add on a U at the end.\\ 
Rule schema: $XI \;\Rightarrow\; XIU$.

\vspace{1em}

\textbf{Rule 2}:
Suppose you have Mx. Then you may add Mxx to your collection.\\
Rule schema: $MX \;\Rightarrow\; MXX$.

\vspace{1em}

\textbf{Rule 3}:
If III occurs in one of the strings in your collection, you may make a new
string with U in place of III.\\
Rule schema: $XIIIY \;\Rightarrow\; XUY$.

\vspace{1em}

\textbf{Rule 4}:
If UU occurs inside one of your strings, you can drop it.\\
Rule schema: $XUUY \;\Rightarrow\; XY$.
\end{quote}

The puzzle does not have a solution.  When experimenting with the rules given the initial condition, you
quickly realize you can only start using rule 1 or rule 2.  With rule 1, you realize once you have $MIU$, there's 
never a branch that allows you to get rid of the $I$ after the $M$, so you look to rule 2, which seems more promising
at first, as you are at least able to transform into strings in the form of $MUX$, which at least contains the $MU$. You might
also try working backwards, trying to get $MIII$ or $MUUU$, as these would both immediately simplify to $MU$. But as
you mess with it more, you see the same problems occurring over and over, where you can't get an odd number of sequential $U's$ without also getting other pieces of string 
between the initial $M$ and the $U's$, and you can't get an odd number of sequential $I's$ at all.  I don't have a proof, but it does seem
as if the rules impose strict limits on the type of strings you can get, so it seems if you can't meet the strings you want to see (that'd lead to MU)
to the strings you are actually able to transform into, then there is no path to the solution.



\subsubsection{Questions}

HW 1 Question: One way the $MU$ puzzle seems impossible is by going backwards, where to get $MU$ it seems like 
you'd have to get $MIII$ or $MUUU$ first, but getting $MIII$ or $MUUU$ feels impossible almost immediately with
the limited rules.  But would this way of looking at it make sense for a larger puzzle with more rules? 
I can imagine a similar puzzle with more rules where getting $MIII$ or $MUUU$ is still 
impossible, but other rules allow you to get to $MU$ without using the $XIIIY \;\Rightarrow\; XUY$ or $XUUY \;\Rightarrow\; XY$ rules.

\subsection{Week 1}

\section{Essay}

\section{Evidence of Participation}

\section{Conclusion}\label{conclusion}

\begin{thebibliography}{99}
\bibitem[BLA]{bla} Author, \href{https://en.wikipedia.org/wiki/LaTeX}{Title}, Publisher, Year.
\end{thebibliography}

\end{document}
